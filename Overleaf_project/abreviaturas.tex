%%%%%%%%%%%%%% Como usar o pacote acronym
% \ac{acronimo} -- Na primeira vez que for citado o acronimo, o nome completo irá aparecer
%                  seguido do acronimo entre parênteses. Na proxima vez somente o acronimo
%                  irá aparecer. Se usou a opção footnote no pacote, entao o nome por extenso
%                  irá aparecer aparecer no rodapé
%
% \acf{acronimo} -- Para aparecer com nome completo + acronimo
% \acs{acronimo} -- Para aparecer somente o acronimo
% \acl{acronimo} -- Nome por extenso somente, sem o acronimo
% \acp{acronimo} -- igual o \ac mas deixando no plural com S (ingles)
% \acfp{acronimo}--
% \acsp{acronimo}--
% \aclp{acronimo}--

\chapter*{Lista de abreviaturas e siglas}%
% \addcontentsline{toc}{chapter}{Lista de abreviaturas e siglas}
\markboth{Lista de abreviaturas e siglas}{}


\begin{acronym}
    \acro{ADR}{\textit{Adaptive Data Rate}}
    \acro{AES}{\textit{Advanced Encryption Standard}}
	\acro{API}{\textit{application programming interface}}
	\acro{CAP}{Período de Acesso com Contenção}
	\acro{CCA}{\textit{Clear Channel Assessment}}
    \acro{CE}{\textit{Control Element}}
	\acro{CFP}{Período de Acesso sem Contenção}
    \acro{CSMA}{\textit{Carrier Sense Multiple Access}}
	\acro{CS-MA/CA}{\textit{Carrier Sense Multiple Access with Collision Avoidance}}
    \acro{CSS}{\textit{chirp spread spectrum}}
	\acro{CTS}{\textit{Clear to Send}}
    \acro{DPID}{\textit{Datapath Identifier}}
	\acro{ED}{\textit{Energy Detection}}
	\acro{XML}{\textit{Extensible Markup Language}}
    \acro{FE}{\textit{Forwarding Element}}
	\acro{FFD}{Dispositivo de Função Completa}
    \acro{ForCES}{\textit{Forwarding and Control Element Separation}}
	\acro{FWD}{Forwarding}
	\acro{GTS}{Compartimento de Tempo Garantido}
    \acro{ID}{identificação}
    \acro{IEEE}{\textit{Institute of Electrical and Electronics Engineers}}
    \acro{IETF}{\textit{Internet Engineering Task Force}}
    \acro{IFSC/SJ}{Instituto Federal de Santa Catarina - Câmpus São José}
	\acro{INPP}{\textit{In-Network Packet Processing}}
	\acro{IoT}{Internet das Coisas}
    \acro{IP}{\textit{Internet Protocol address}}
	\acro{ISM}{Industrial, Cientifico e Médico}
	\acro{LFB}{\textit{Logical Function Block}}
	\acro{LPWAN}{\textit{Low Power Wide Area Network}}
	\acro{LQI}{\textit{Link Quality Indication}}
	\acro{LR-WPAN}{\textit{Low-Rate Wireless Personal Area Network}}
	\acro{MAC}{Controle de Acesso ao Meio}
	\acro{MANET}{Rede \textit{Ad hoc} móvel}
	\acro{PAN}{\textit{Personal Area Network}}
	\acro{PHY}{física}
	\acro{QoS}{Quality of Service}
	\acro{RAM}{\textit{Random Access Memory}}
	\acro{RFD}{Dispositivo de Função Reduzida}
	\acro{ROM}{\textit{Read Only Memory}}
	\acro{RSMSF}{Rede de Sensores Multimídia sem Fio}
	\acro{RSSF}{Rede de Sensores Sem Fio}
	\acro{RSSI}{\textit{Received signal strength indication}}
	\acro{RTS}{\textit{Request to Send}}
	\acro{SDN}{Redes Definidas por Software}
	\acro{SDN-WISE}{\textit{Software Defined Networking for WIreless SEnsor networks}}
	\acro{SNR}{relação sinal-ruído}
	\acro{TCP}{\textit{Transmission Control Protocol}}
	\acro{TD}{\textit{Topology Discovery}}
	\acro{TTL}{\textit{Time To Live}}
\end{acronym}

